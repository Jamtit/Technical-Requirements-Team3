% Kompiuterijos katedros ir kibernetinio saugumo laboratorijos šablonas
% Template of Department of Computer Science II or cybersecurity laboratory
% Versija 1.3 2021 m. birželis [ March, 2015]

\documentclass[a4paper,12pt,fleqn]{article}
\input{allPacks}

\usepackage{graphicx}
\graphicspath{{./images/}}

\newtoggle{inLithuanian}
 %If the report is in Lithuanian, it is set to true; otherwise, change to false
\settoggle{inLithuanian}{false}

%create file preface.tex for the preface text
%if preface is needed set to true
\newtoggle{needPreface}
\settoggle{needPreface}{false}

\newtoggle{signaturesOnTitlePage}
\settoggle{signaturesOnTitlePage}{false}

\input{macros}

\begin{document}
 % #1 -report type, #2 - title, #3-7 students, #8 - supervisor
 \depttitlepage{Information Technology II year}{Technical Specification\\{\small Software Engineering Project | Team 3}}{Titas Majauskas} 
 {Dinas Majauskas}{Jomantas Užusinas}{Vilius Juknevičius}{Sakalas Stasiulis}% students 2-5
 {Virgilijus Krinickij, Gediminas Rimša}

\tableofcontents


 %Introduction section: label is sec:intro
\newpage
\section{Purpose}

\subsection{Purpose of The Document}
The purpose of this document is to provide an overview of the whole system, while visualizing and explaining the main parts of a system via diagrams, technological decisions, design sketches and verbal explanations. 

\section{Diagrams}
\begin{center}
    \includegraphics[scale=0.4]{main/images/system-context-diagram.png}
\end{center}

\subsection{System-context diagram}
A \textbf{system context diagram}, shows \textbf{three} external objects that interact
with the system. These objects have either arrows pointing to them or from them.\\ Arrow meaning:
\begin{itemize}
    \item An arrow line that points \textit{from the system to an object} - shows
what the \textbf{system provides} to that object.
    \item An arrow line that points \textit{from an object to the
system} - shows what that \textbf{object provides} to the system.
\end{itemize}
Objects meanings and purpose:
\begin{itemize}
    \item \textbf{Team 1} - will only act as a provider to the system. It will supply Area 3 with the necessary data about the roof, most likely in JSON format as primary format style.
    \item \textbf{Bpypolyskel} - will only act as a provider to the system. It will supply the whole solution with functionalities needed to compile a 3D model of the given roof and provide a visualisation of that model.
    \item \textbf{Graphics.py} - will only act as a provider to the system. It will provide the system with functions to create a simple 2D model of said roof using the basic data.
    \item \textbf{Team 4} - will act as a receiver of our library. Team 4 will receive the roof data that will be compiled through our algorithms. The format of the data sent to this team will be JSON as primary format.
\end{itemize}

\includegraphics[scale=0.4]{main/images/use-case.png}

\newpage
\section{Technological Decisions}
The library will be built using \textbf{Python} programming language in \textbf{Visual Studio Code IDE}. The library will be accessible to everyone through Python \textit{preferred installer program} (PIP). Its code will use \textbf{Git} version control, and all commits will be archived in a designated repository.\newline
For testing purposes our team will be using a Python testing framework called \textbf{Pytest}. 

\subsection{Visual Studio Code}
Visual Studio Code is a streamlined code editor with support for development operations like debugging, task running, and version control.

\subsection{Python}
Python is used for web and software development, data analytics, machine learning, and even design. Python is a programming language that will be used to implement required solution. We will use Python 3.11.0 - the newest current version available (2022-10-25).

\subsection{Pytest/Pytest-Snapshot}


\begin{itemize}
    \item \textbf{Pytest} - a testing framework, which makes it easy to write small, readable tests, and can scale to support complex functional testing for applications and libraries.
    \item \textbf{Pytest-Snapshot} - a plugin for snaphshot testing with Pytest.
    Snapshot testing can be used to test that the value of an expression does not change unexpectedly.
\end{itemize}

\subsection{Python preferred installer program}
\textit{Python preferred installer program (PIP)} is used for installing libraries for Python.
\begin{itemize}
    \item \textbf{NumPy} - a Python library used for working with arrays. It also has functions for working in domain of linear algebra and matrices.
    \item \textbf{Matplotlib} - a comprehensive library for creating static, animated, and interactive visualizations in Python.
    \item \textbf{Mathutils} - a general math utilities library providing Matrix, Vector, Quaternion, Euler and Color classes.
    \item \textbf{Graphics.py} - a library used to draw simple mathematical figures, using two-dimensional vector points
\end{itemize}


\newpage
\section{Testing}
Given the testing framework our team will use, testing will be done on these said sections of our solution:
\begin{itemize}
    \item \textbf{Position:}
    \begin{itemize}
        \item Do testing to check if the position of the solar panel will not be crossed with position of chimneys, skylights, etc.
        \item Do testing to check if solar panels are not put over the roof's borders.
    \end{itemize}
    \item \textbf{Coverage}
    \begin{itemize}
        \item Do testing to check if solar panel covers the maximum possible area of the roof having in mind many limitations.
    \end{itemize}
\end{itemize}
For later, more complex testing \textbf{Pytest} plugin \textbf{Pytest-Snapshot} will be used to check if some changes happened while compiling the needed data.

\end{document}
